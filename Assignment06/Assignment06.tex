\documentclass{article}
\usepackage[top=1in, bottom=1in, left=1in, right=1in]{geometry}
% \usepackage{fullpage, fancyhdr}
\usepackage{fullpage}
\usepackage{float}
\usepackage{mathtools}
\usepackage{xfrac}
\usepackage{graphicx}
\usepackage{caption}
\usepackage{subcaption}
\usepackage{portland}
%\usepackage{setspace}
\setlength{\topmargin}{0.0in}
\setlength{\headheight}{0.5in}
\setlength{\headsep}{0in}
\setlength{\footskip}{9pt}
\usepackage{listings}
\usepackage{color}

\usepackage{multicol}

\renewcommand{\arraystretch}{1.5}

% For circuitikz
\usepackage[american,arrowmos]{circuitikz}
\usepackage{tikz}
\usetikzlibrary{calc}
\usepackage{pgfplots}
\usepackage{amsfonts}
\usetikzlibrary{shapes,arrows}

% \pagestyle{fancyplain}
\pagestyle{myheadings}
\voffset=-0.50in
\topmargin=0.00in 
\headsep=0.25in 
\evensidemargin=0in 
\oddsidemargin=0in 
\textwidth=6.6in 
\textheight=10.0in 

\renewcommand{\topfraction}{0.9}	% max fraction of floats at top
\renewcommand{\bottomfraction}{0.8}	% max fraction of floats at bottom
%   Parameters for TEXT pages (not float pages):
\setcounter{topnumber}{2}
\setcounter{bottomnumber}{2}
\setcounter{totalnumber}{4}     % 2 may work better
\setcounter{dbltopnumber}{2}    % for 2-column pages
\renewcommand{\dbltopfraction}{0.9}	% fit big float above 2-col. text
\renewcommand{\textfraction}{0.07}	% allow minimal text w. figs
%   Parameters for FLOAT pages (not text pages):
\renewcommand{\floatpagefraction}{0.7}	% require fuller float pages
% N.B.: floatpagefraction MUST be less than topfraction !!
\renewcommand{\dblfloatpagefraction}{0.7}	% require fuller float pages
% remember to use [htp] or [htpb] for placement

\title{Assignment \# 6: MEMS PPA and Doppler Calculations}
\date{3/27/2013}
\author{Brian Arnberg}

\markright{Brian Arnberg\hfill ELEC 6760 - Solid State Sensors\hfill}     
\setlength{\parindent}{0pt}


\begin{document}\label{start}

% \begin{titlepage}
% 	\maketitle
% 	\thispagestyle{empty}
% \end{titlepage}


\section*{ Homework Assignment \#6 - Due Wed. 3/27/13 }
\renewcommand{\labelenumi}{\arabic{enumi})}
\begin{multicols}{2}
\begin{enumerate}
%--------------------------------------------------------------%
%---- Problem 1 -----------------------------------------------%
\item\label{p1}
 A parallel plate actuator (PPA) consists of two square electrodes, 500$\mu$m on a
     side, separated by 10$\mu$m, in a vacuum. What is the force produced if 100V DC is
     applied across the electrodes

		\begin{tabular}{ l }
			$F = \frac{\varepsilon_0 \varepsilon_r A V^2}{2 x^2}$\\
			$F = \frac{8.854e-12*0*(500\mu m)^2*100^2}{2*(10\mu m)^2}$\\
			$F = 110.6 \mu N$
		\end{tabular}
%--------------------------------------------------------------%
%---- Problem 2 -----------------------------------------------%
\item\label{p2}
     If a sinusoidal voltage with a 100V amplitude is applied to the PPA in (\ref{p1}), where
     the frequency of the voltage signal is much higher than the natural frequency of
     the mechanical system, what is the average force produced by the PPA?

		\begin{tabular}{ l }
			$V^2(t) \approx {V_s}^2/2 = 100^2/2 = 5000V RMS$\\
			$F = \frac{\varepsilon_0 \varepsilon_r A V^2}{2 x^2}$\\
			$F = \frac{8.854e-12*0*(500\mu m)^2*5000}{2*(10\mu m)^2}$\\
			$F = 55.33 \mu N$
		\end{tabular}
  
%--------------------------------------------------------------%
%---- Problem 3 -----------------------------------------------%
\item\label{p3}
     For the PPA in (\ref{p1}), if the system spring constant is 50N/m, what is the pull-in
     voltage?

		\begin{tabular}{ l }
			$V_p = \sqrt{\frac{8*k*{x_0}^3}{27*A*\varepsilon_r\varepsilon_0}}$\\
			$V_p = \sqrt{\frac{8*50*{10e-6}^3}{27*(500e-6)^2*8.854e-12}}$\\
			$V_p = 81.8V$
		\end{tabular}
%--------------------------------------------------------------%
%---- Problem 4 -----------------------------------------------%
\item\label{p4}
	If the PPA in (\ref{p1}) is used with the spring in (\ref{p3}), what applied DC voltage will
     decrease the distance between the electrodes by 1$\mu$m.

		\begin{tabular}{ l }
			$F = \frac{\varepsilon_0 \varepsilon_r A V^2}{2 (x_0 - x(t))^2} = k x(t)$\\
			$\frac{8.854e-12*0*(500\mu m)^2*V^2}{2*(10\mu m - 1\mu m)^2} = 50 N/m * 1\mu m$\\
			$V^2 = \frac{8.1e-15}{2.2135e-18} = 3.659e3$\\
			$V = 60.49 V$
		\end{tabular}
  
%--------------------------------------------------------------%
%---- Problem 5 -----------------------------------------------%
\item\label{p5}
     An object is moving away from a 20KHz sound source at 10m/s. If the speed of
     sound in air is 331m/s, what is the frequency of the reflected sound wave?

		\begin{tabular}{ l }
			$f_d = \frac{f_s}{1+\dot{x}/c}$\\
			$f_d = \frac{20kHz}{1+10/331}$\\
			$f_d = 19.41kHz$
		\end{tabular}
  
%--------------------------------------------------------------%
%---- Problem 6 -----------------------------------------------%
\item\label{p6}
	If the object in (\ref{p5}) is moving toward the sound source, what is the frequency of
     the return sound wave?
% \begin{figure}[h]
% \centering

		\begin{tabular}{ l }
			$f_d = \frac{f_s}{1+\dot{x}/c}$\\
			$f_d = \frac{20kHz}{1-10/331}$\\
			$f_d = 20.62kHz$
		\end{tabular}
% \end{figure}

\end{enumerate}
\end{multicols}
\label{end}\end{document}


