\documentclass{article}
\usepackage[top=1in, bottom=1in, left=1in, right=1in]{geometry}
% \usepackage{fullpage, fancyhdr}
\usepackage{fullpage}
\usepackage{float}
\usepackage{mathtools}
\usepackage{xfrac}
\usepackage{graphicx}
\usepackage{caption}
\usepackage{subcaption}
\usepackage{portland}
%\usepackage{setspace}
\setlength{\topmargin}{0.0in}
\setlength{\headheight}{0.5in}
\setlength{\headsep}{0in}
\setlength{\footskip}{9pt}
\usepackage{listings}
\usepackage{color}

\renewcommand{\arraystretch}{1.5}

% For circuitikz
\usepackage[american,arrowmos]{circuitikz}
\usepackage{tikz}
\usetikzlibrary{calc}
\usepackage{pgfplots}
\usepackage{amsfonts}
\usetikzlibrary{shapes,arrows}

% \pagestyle{fancyplain}
\pagestyle{myheadings}
\voffset=-0.50in
\topmargin=0.00in 
\headsep=0.25in 
\evensidemargin=0in 
\oddsidemargin=0in 
\textwidth=6.6in 
\textheight=10.0in 

\renewcommand{\topfraction}{0.9}	% max fraction of floats at top
\renewcommand{\bottomfraction}{0.8}	% max fraction of floats at bottom
%   Parameters for TEXT pages (not float pages):
\setcounter{topnumber}{2}
\setcounter{bottomnumber}{2}
\setcounter{totalnumber}{4}     % 2 may work better
\setcounter{dbltopnumber}{2}    % for 2-column pages
\renewcommand{\dbltopfraction}{0.9}	% fit big float above 2-col. text
\renewcommand{\textfraction}{0.07}	% allow minimal text w. figs
%   Parameters for FLOAT pages (not text pages):
\renewcommand{\floatpagefraction}{0.7}	% require fuller float pages
% N.B.: floatpagefraction MUST be less than topfraction !!
\renewcommand{\dblfloatpagefraction}{0.7}	% require fuller float pages
% remember to use [htp] or [htpb] for placement

\title{Assignment \# 5: MEMS Capacitive Sensor Calculations}
\date{2/20/2013}
\author{Brian Arnberg}

\markright{Brian Arnberg\hfill ELEC 6760 - Solid State Sensors\hfill}     
\setlength{\parindent}{0pt}


\begin{document}\label{start}

% \begin{titlepage}
% 	\maketitle
% 	\thispagestyle{empty}
% \end{titlepage}


\section*{ Homework Assignment \#5 - Due Wed. 2/20/13 }
\renewcommand{\labelenumi}{\arabic{enumi})}

\begin{enumerate}
%--------------------------------------------------------------%
%---- Problem 1 -----------------------------------------------%
\item\label{p1}
 A MEMS device consists of a proof mass attached to a frame with a suspension
     system. The bottom of the proof mass is 1mm by 1mm in size and serves as an
     electrode. Another electrode of the same size is located 2$\mu$m beneath it. If the
     proof mass can move up and down $\pm 1\mu$m from its nominal distance to the bottom
     electrodes, calculate the nominal, maximum and minimum capacitance between
     the two electrodes. Assume that the device is in a vacuum.\\

\begin{figure}[h]
\centering
\caption{ Values of Capacitance for (\ref{p1}).\\
		$d = 2\mu m \colon \Delta d = \pm1\mu m \colon \varepsilon = 8.854 pF/m$}
	\begin{subtable}[b]{0.3\textwidth}
		\centering
		\begin{tabular}{ l }
			$C = \frac{\varepsilon \times w \times L}{d - \Delta d}$\\
			$C = \frac{8.854 pF/m \times 1mm \times 1mm}{2\mu m - 1\mu m}$\\
			$C = 8.854pF$
		\end{tabular}
	  	\caption{Maximum: $d = d - \Delta d$}
	\end{subtable}
	\begin{subtable}[b]{0.3\textwidth}
		\centering
		\begin{tabular}{ l }
			$C = \frac{\varepsilon \times w \times L}{d}$\\
			$C = \frac{8.854 pF/m \times 1mm \times 1mm}{2\mu m}$\\
			$C = 4.427pF$
		\end{tabular}
		\caption{Nominal: $d = d$} 
	\end{subtable}
	\begin{subtable}[b]{0.3\textwidth}
		\centering
		\begin{tabular}{ l }
			$C = \frac{\varepsilon \times w \times L}{d - \Delta d}$\\
			$C = \frac{8.854 pF/m \times 1mm \times 1mm}{2\mu m + 1\mu m}$\\
			$C = 2.951pF$
		\end{tabular}
		\caption{Minimum: $d = d + \Delta d$}
	\end{subtable}
\end{figure}
%--------------------------------------------------------------%
%---- Problem 2 -----------------------------------------------%
\item\label{p2}
A certain MEMS capacitance has a rest (i.e. nominal) value of 3pF, a minimum
     value of 2pF and a maximum value of 5pF. Place it in a charge amplifier circuit
     that has an input voltage of 10V and a feedback capacitor (C2) of 10pF. Calculate
     the amplifier output voltage (at the end of the $\phi_2$ cycle) for the nominal, minimum
     and maximum capacitance values.\\

\begin{figure}[h]
\centering
\caption{Amplifier Output Voltages for (\ref{p2}).\\
$V_{in} = 10V \colon C_2 = 10pF \colon \text{After }\phi_2\text{, }V_{out} = \frac{-C_s V_{in}}{C_2}$}
	\begin{subtable}[b]{0.3\textwidth}
		\centering
		\begin{tabular}{ l }
			$C_s = 2pF$\\
			$V_{out} = \frac{-C_s V_{in}}{C_2} = \frac{-2pF \times 10V}{10pF}$\\
			$V_{out} = -2V$
		\end{tabular}
		\caption{$V_{out}$ for $C_{min}$}
	\end{subtable}
	\begin{subtable}[b]{0.3\textwidth}
		\centering
		\begin{tabular}{ l }
			$C_s = 3pF$\\
			$V_{out} = \frac{-C_s V_{in}}{C_2} = \frac{-3pF \times 10V}{10pF}$\\
			$V_{out} = -3V$
		\end{tabular}
		\caption{$V_{out}$ for $C_{nom}$}
	\end{subtable}
	\begin{subtable}[b]{0.3\textwidth}
		\centering
		\begin{tabular}{ l }
			$C_s = 5pF$\\
			$V_{out} = \frac{-C_s V_{in}}{C_2} = \frac{-3pF \times 10V}{10pF}$\\
			$V_{out} = -5V$
		\end{tabular}
		\caption{$V_{out}$ for $C_{max}$}
	\end{subtable}
\end{figure}


%--------------------------------------------------------------%
%---- Problem 3 -----------------------------------------------%
\item\label{p3}
	For the MEMS capacitance in (\ref{p2}) place it in a 5V ``fast'' CMOS ring oscillator
	circuit with both resistors being $100k\Omega$. What is the output frequency for $C_{min}$,
	$C_{nom}$ and $C_{max}$?\\

\begin{figure}[h]     
\centering
\caption{Output Frequencies for (\ref{p3}).\\
		$t = -RC \ln {\sfrac{1}{3}} \colon $
		$f = 1/_T \colon T = 2 t \colon f = \frac{0.455}{RC}$}
	\begin{subtable}[b]{0.3\textwidth}
		\centering
		\begin{tabular}{ l }
			$C = 2pF \colon R = 100k\Omega$\\
			$f = \frac{0.455}{RC} = \frac{0.455}{100k\Omega 2pF}$\\
			$f = 2.275$MHz
		\end{tabular}
	  	\caption{$f_o$ for $C_{min}$}
	\end{subtable}
	\begin{subtable}[b]{0.3\textwidth}
		\centering
		\begin{tabular}{ l }
			$C = 3pF \colon R = 100k\Omega$\\
			$f = \frac{0.455}{RC} = \frac{0.455}{100k\Omega 3pF}$\\
			$f = 1.517$MHz
		\end{tabular}
		\caption{$f_o$ for $C_{nom}$} 
	\end{subtable}
	\begin{subtable}[b]{0.3\textwidth}
		\centering
		\begin{tabular}{ l }
			$C = 5pF \colon R = 100k\Omega$\\
			$f = \frac{0.455}{RC} = \frac{0.455}{100k\Omega 5pF}$\\
			$f = 910$kHz
		\end{tabular}
		\caption{$f_o$ for $C_{max}$}
	\end{subtable}

\end{figure}


\newpage
%--------------------------------------------------------------%
%---- Problem 4 -----------------------------------------------%
\item\label{p4}
If two MEMS capacitances from (\ref{p2}) are placed in a capacitive AC voltage divider
     to realize a differential capacitive sensor configuration, with the input voltage
     having an amplitude of 10V, what is the output voltage amplitude for each case?
\begin{figure}[h]
\centering
\caption{Output Voltage Amplitude for each case in (\ref{p4}).}
\begin{subfigure}[b]{0.3\textwidth}
	\centering
	\begin{circuitikz}[scale=0.9]\draw
		(0,1) node[ground] (gnd) {}
			to[vC, l_=$C_2$, o-*] (0,3)
		        to[vC, l_=$C_1$, *-o]  (0,5)
			node[anchor=south] {$+10V$};
		\draw (0,3) to[short, *-*] (2,3)
			node[anchor=south] {$V_{out}$};
	\end{circuitikz}
	\caption{AC Voltage Divider}
\end{subfigure}
\begin{subtable}[b]{0.3\textwidth}
	\centering
	\begin{tabular}{ l }
		$Q = VC \colon C_{eq} = \frac{C_2 C_1}{C_1 + C_2} \colon$\\
		$V_{out} = \frac{Q}{C_2} = \frac{V_in C_{eq}}{C_2}$\\
		$V_{out} = V_{in} \frac{C_1}{C_1 + C_2}$
	\end{tabular}
	\caption{Calculations}
\end{subtable}
\begin{subtable}[b]{0.3\textwidth}
	\centering
	\begin{tabular}{|c|c|c|}
	\hline
	\multicolumn{1}{|l|}{$C_1$, pF} & \multicolumn{1}{l|}{$C_2$, pF} &%
	        \multicolumn{1}{l|}{$V_{out}$, V} \\ \hline
		2 (min) & 5 (max) & 2.86 \\ \hline
		3 (nom) & 3 (nom) & 5.00 \\ \hline
		5 (max) & 2 (min) & 7.14 \\ \hline
	\end{tabular}
	\caption{Voltage outputs for each case.}
\end{subtable}
\end{figure}
%--------------------------------------------------------------%
%---- Problem 5 -----------------------------------------------%
\item\label{p5}
If the MEMS capacitance from (\ref{p2}) is placed in a switched-capacitor circuit that is
     switched at 250KHz, what is the value of the equivalent resistance for the
     nominal, minimum and maximum capacitance values?\\
\begin{figure}[h]     
\centering
\caption{Equivalent Resistances for (\ref{p5}).}
	\begin{subfigure}[b]{0.5\textwidth}
		\centering	
		\begin{tabular}{ c }
			$f_{switch} = 250kHz$\\
			$i = \frac{\delta q}{\delta t} = \frac{Q}{T}$%
			$  = \frac{C V_{in}}{T} = C f_{switch} V_{in}$\\
			$\Rightarrow R = \frac{V}{i} = \frac{1}{C f_{switch}}$\\
			\hline
		\end{tabular}
	\end{subfigure}

	\begin{subtable}[b]{0.3\textwidth}
		\centering
		\begin{tabular}{ l }
			$C = 2pF$\\
			$R = \frac{1}{C f_{switch}} = \frac{1}{2pF \times 250kHz}$\\
			$R = 2M\Omega$
		\end{tabular}
	  	\caption{$R$ for $C_{min}$}
	\end{subtable}
	\begin{subtable}[b]{0.3\textwidth}
		\centering
		\begin{tabular}{ l }
			$C = 3pF$\\
			$R = \frac{1}{C f_{switch}} = \frac{1}{3pF \times 250kHz}$\\
			$R = 1.333M\Omega$
		\end{tabular}
		\caption{$R$ for $C_{nom}$} 
	\end{subtable}
	\begin{subtable}[b]{0.3\textwidth}
		\centering
		\begin{tabular}{ l }
			$C = 5pF$\\
			$R = \frac{1}{C f_{switch}} = \frac{1}{5pF \times 250kHz}$\\
			$R = 800k\Omega$
		\end{tabular}
		\caption{$R$ for $C_{max}$}
	\end{subtable}

\end{figure}

%--------------------------------------------------------------%
%---- Problem 6 -----------------------------------------------%
\item\label{p6}
If the MEMS device in (\ref{p2}) is placed in an RC phase delay circuit, where R = 250k$\Omega$, 
	what is the phase delay in $\mu s$ for the nominal, minimum and maximum
	capacitance values?\\

\begin{figure}[h]     
\centering
\caption{Phase Delay for (\ref{p6}).}
	\begin{subfigure}[b]{0.5\textwidth}% initial calculations
		\centering	
		\begin{tabular}{ r }
			$V_{out} = \frac{V_{in}}{s}\frac{1/{s C}}{R + 1/{s C}}$%
			$ = \frac{V_{in}}{s} - \frac{V_{in}}{s + 1/{RC}}$\\
			$V_{out}(t) = V_{in}(1 - e^{-t/{RC}})$\\
			$t \colon V_{out} = V_{in}/_2 \rightarrow V_{in}/_2 = V_{in} (1 - e^{-t/{RC}})$\\
			$1/_2 = 1 - e^{-t/{RC}} \rightarrow t = -RC \ln \sfrac{1}{2}$\\
			$t = 0.693 R C$\\
			\hline
		\end{tabular}
	\end{subfigure}

	\begin{subtable}[b]{0.3\textwidth}
		\centering
		\begin{tabular}{ l }
			$C = 2pF \colon R = 250k\Omega$\\
			$t = 0.692 (2pf)(250k\Omega)$\\
			$t = 0.346\mu s$
		\end{tabular}
	  	\caption{Phase delay for $C_{min}$}
	\end{subtable}
	\begin{subtable}[b]{0.3\textwidth}
		\centering
		\begin{tabular}{ l }
			$C = 3pF \colon R = 250k\Omega$\\
			$t = 0.692 (3pf)(250k\Omega)$\\
			$t = 0.520\mu s$
		\end{tabular}
		\caption{Phase delay for $C_{nom}$} 
	\end{subtable}
	\begin{subtable}[b]{0.3\textwidth}
		\centering
		\begin{tabular}{ l }
			$C = 5pF \colon R = 250k\Omega$\\
			$t = 0.692 (5pf)(250k\Omega)$\\
			$t = 0.866\mu s$
		\end{tabular}
		\caption{Phase delay for $C_{max}$}
	\end{subtable}

\end{figure}

\end{enumerate}
\label{end}\end{document}


