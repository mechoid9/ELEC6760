\documentclass{article}
\usepackage[top=1in, bottom=1in, left=1in, right=1in]{geometry}
% \usepackage{fullpage, fancyhdr}
\usepackage{fullpage}
\usepackage{float}
\usepackage{mathtools}
\usepackage{graphicx}
\usepackage{caption}
\usepackage{subcaption}
\usepackage{portland}
%\usepackage{setspace}
\setlength{\topmargin}{0.0in}
\setlength{\headheight}{0.5in}
\setlength{\headsep}{0in}
\setlength{\footskip}{9pt}
\usepackage{listings}
\usepackage{color}

\renewcommand{\arraystretch}{1.5}

% For circuitikz
\usepackage[american,arrowmos]{circuitikz}
\usepackage{tikz}
\usetikzlibrary{calc}
\usepackage{pgfplots}
\usepackage{amsfonts}
\usetikzlibrary{shapes,arrows}

% \pagestyle{fancyplain}
\pagestyle{myheadings}
\voffset=-0.50in
\topmargin=0.00in 
\headsep=0.25in 
\evensidemargin=0in 
\oddsidemargin=0in 
\textwidth=6.6in 
\textheight=10.0in 

\renewcommand{\topfraction}{0.9}	% max fraction of floats at top
\renewcommand{\bottomfraction}{0.8}	% max fraction of floats at bottom
%   Parameters for TEXT pages (not float pages):
\setcounter{topnumber}{2}
\setcounter{bottomnumber}{2}
\setcounter{totalnumber}{4}     % 2 may work better
\setcounter{dbltopnumber}{2}    % for 2-column pages
\renewcommand{\dbltopfraction}{0.9}	% fit big float above 2-col. text
\renewcommand{\textfraction}{0.07}	% allow minimal text w. figs
%   Parameters for FLOAT pages (not text pages):
\renewcommand{\floatpagefraction}{0.7}	% require fuller float pages
% N.B.: floatpagefraction MUST be less than topfraction !!
\renewcommand{\dblfloatpagefraction}{0.7}	% require fuller float pages
% remember to use [htp] or [htpb] for placement

\title{Assignment \# 4: Resistor Calculations}
\date{2/11/2013}
\author{Brian Arnberg}

\markright{Brian Arnberg\hfill ELEC 6760 - Solid State Sensors\hfill}     
\setlength{\parindent}{0pt}


\begin{document}\label{start}

% \begin{titlepage}
% 	\maketitle
% 	\thispagestyle{empty}
% \end{titlepage}


\section*{ Homework Assignment \#4 - Due Fri. 2/15/13 }
\renewcommand{\labelenumi}{\arabic{enumi})}

\begin{enumerate}
%--------------------------------------------------------------%
%---- Problem 1 -----------------------------------------------%
\item An unbuffered resistive sensor, $R_s$, is in the circuit shown below, where $500\Omega \le R_s \le 1.5k\Omega$. What is $V_{out}$ for the minimum, mid range and maximum resistance values of the sensor?\\
\begin{figure}[h]
\centering
	\begin{subfigure}[b]{0.3\textwidth}
	\centering
	\begin{circuitikz}[scale=0.9]\draw
		(0,1) node[anchor=north] {$-5V$}
			to[vR=$R_2$, o-*] (0,3)
		        to[R, l=$R_1$, *-o]  (0,5)
			node[anchor=south] {$+5V$};
		\draw (0,3) to[short, *-*] (2,3)
			node[anchor=south] {$V_{out}$};
		\draw (0,1) to[vR, l_=$R_s$] (0,3);
		\draw (0,3) to[R, l_=$1k\Omega$]  (0,5);
	\end{circuitikz}
	\caption{The Circuit}
	\end{subfigure}
	\begin{subtable}[b]{0.3\textwidth}
		\centering
		\begin{tabular}{ l }
		    $V_{out} = -5 + \frac{10 R_2}{R_2 + R_1}$\\
		\end{tabular}
		\caption{Calculations}
	\end{subtable}
	\begin{subtable}[b]{0.3\textwidth}
	\centering
	\begin{tabular}{  r | r  }
		  $R_s, \Omega$ & $V_{out}, V$ \\
		\hline 
		500 & -1.67\\
		1k & 0\\
		  1.5k & 1 \\
	\end{tabular}
	\caption{Voltage vs. Resistance}
	\end{subtable}
\end{figure}
%--------------------------------------------------------------%
%---- Problem 2 -----------------------------------------------%
\item For the differential resistance sensor shown below, where $R_1 = 1k\Omega + \Delta R$, $R_2 = 1k\Omega - \Delta R$,\\ and $0\Omega \le \Delta R \le 100\Omega$, calculate the minimum and maximum $V_{out}$.
\begin{figure}[h]
\centering
\begin{subfigure}[b]{0.3\textwidth}
	\centering
	\begin{circuitikz}[scale=0.9]\draw
		(0,1) node[ground] (gnd) {}
			to[vR, l_=$R_2$, o-*] (0,3)
		        to[vR, l_=$R_1$, *-o]  (0,5)
			node[anchor=south] {$+10V$};
		\draw (0,3) to[short, *-*] (2,3)
			node[anchor=south] {$V_{out}$};
	\end{circuitikz}
	\caption{The Circuit}
\end{subfigure}
\begin{subtable}[b]{0.3\textwidth}
	\centering
	\begin{tabular}{ l }
		$V_{out} = \frac{10 R_2}{R_2 + R_1} = \frac{10(1k\Omega - \Delta R)}{1k\Omega - \Delta R + 1k\Omega +\Delta R)}$\\
		$V_{out} = \frac{10(1k\Omega) - 10\Delta R}{2(1k\Omega)}$\\
		$V_{out} = 5 - 5\frac{\Delta R}{1k\Omega}$\\
	\end{tabular}
	\caption{Calculations}
\end{subtable}
\begin{subtable}[b]{0.3\textwidth}
	\centering
	\begin{tabular}{ r | r || l }
		$\Delta R$ & $V_{out}$ \\
		\hline
		$0\Omega$ & 5V & max \\
		$100\Omega$ & 4.5V & min \\
	\end{tabular}
	\caption{Voltage vs. Resistance}
\end{subtable}


\end{figure}
%--------------------------------------------------------------%
%---- Problem 3 -----------------------------------------------%
\item A rectangular resistive temperature sensor (5mm long, $50\mu$m wide, and $1\mu$m thick), where current flows through the length of the sensor, is made of a material with a resistivity of 
$5 \times 10^{-6}$  $\Omega$-cm at $0^{\circ}C$, and a TCR of $5\times10^{-3} (^{\circ}C)^{-1}$. What is the approximate resistance at $0^{\circ}C$ and $100^{\circ}C$?\\
$l = 5mm\colon 2=50\mu m\colon t = 1\mu m\colon \rho_0 = 5\times10^{-6}\Omega\text{-cm} |  0^{\circ}C\colon TCR = 5\times10^{-3} (^{\circ}C)^{-1} = \alpha$\\
$R = \rho \frac{L}{S} \colon \rho = \rho_0 (1 + \alpha T )$\\
\begin{description}
\item[$T = 0^{\circ} |$] $R  = \rho_0 ( 1 + \alpha T)\frac{l}{w t} = (5\times10^{-6}\Omega\text{-cm}\frac{m}{100cm})(1 + \frac{5\times10^{-3}}{^{\circ}C} 0^{\circ}C)\frac{5\times10^{-3}m}{50\times10^{-6}m 1\times10^{-6}m} = 5\Omega$
\item[$T = 100^{\circ} |$] $R  = \rho_0 ( 1 + \alpha T)\frac{l}{w t} = (5\times10^{-6}\Omega\text{-cm}\frac{m}{100cm})(1 + \frac{5\times10^{-3}}{^{\circ}C} 100^{\circ}C)\frac{5\times10^{-3}m}{50\times10^{-6}m 1\times10^{-6}m} = 7.5\Omega$
	\end{description}

%--------------------------------------------------------------%
%---- Problem 4 -----------------------------------------------%
\item A certain metal strain gauge has a nominal resistance of $10k\Omega$ and a gauge factor of 1.8. If it experiences a 1\% axial strain, what does the resistance become?\\
	$GF = \frac{\delta R}{\delta L} \Rightarrow \Delta R = GF \frac{\Delta L}{L}R$\\
	$\Delta R = 1.8 (0.01) (10k\Omega) = 180\Omega$\\
	$R_{new} = R + \Delta R = 10k + 180 = 10.18k\Omega$
%--------------------------------------------------------------%
%---- Problem 5 -----------------------------------------------%
\item If the strain gauge in (4) experiences a -1\% axial strain, what does the resistance become?\\
	$GF = \frac{\delta R}{\delta L} \Rightarrow \Delta R = GF \frac{\Delta L}{L}R$\\
	$\Delta R = 1.8 (-0.01) (10k\Omega) = -180\Omega$\\
	$R_{new} = R + \Delta R = 10k - 180 = 9.82k\Omega$
%--------------------------------------------------------------%
%---- Problem 6 -----------------------------------------------%
	\newpage
\item A polysilicon differential piezoresistive sensor is connected to a 10V source as shown below, where $R_n$ is a N-type piezoresistor and $R_p$ is a P-Type piezoresistor. With no strain on the piezoresistors, $R_n = R_p = 1k\Omega$. Calculate $V_{out}$ for no strain. If $R_n$ has a GF of -30 and $R_p$ has a GF of +30, and both piezoresistors experience a 0.2\% axial strain, calculate $V_{out}$. 
\begin{figure}[h]
	\centering
\begin{subfigure}[b]{0.2\textwidth}
	\centering
	\begin{circuitikz}[scale=0.9]\draw
		(0,1) node[ground] (gnd) {}
			to[vR, l_=$R_p$, o-*] (0,3)
		        to[vR, l_=$R_n$, *-o]  (0,5)
			node[anchor=south] {$+10V$};
		\draw (0,3) to[short, *-*] (2,3)
			node[anchor=south] {$V_{out}$};
	\end{circuitikz}
	\caption{The Circuit}
\end{subfigure}
\begin{subtable}[b]{0.35\textwidth}
	\centering
	\begin{tabular}{ l }
		$V_{out} = 10\frac{R_p}{R_p + R_n} $\\
		$V_{out} = 10\frac{1k}{1k+1k}$\\
		$V_{out} = 5V$\\
	\end{tabular}
	\caption{Calculation of $V_{out}$ with no strain.}
\end{subtable}
\begin{subtable}[b]{0.35\textwidth}
	\centering
	\begin{tabular}{ l }
		$R_n = R_n + (GF_n)(\varepsilon_l)(R_n)$\\
		$R_n = 1k - 30(.002)(1k) = 940\Omega$\\
		$R_p = R_p + (GF_p)(\varepsilon_l)(R_p)$\\
		$R_p = 1k + 30(.002)(1k) = 1060\Omega$\\
		$V_{out} = \frac{10 R_p}{R_p + R_n}$\\
		$V_{out} = \frac{10 (1060)}{1060+940}$\\
		$V_{out} = 5.3V$\\
	\end{tabular}
	\caption{Calculation of $V_{out}$ with an axial strain of 0.2\%.}
\end{subtable}
\end{figure}
%--------------------------------------------------------------%
%---- Problem 7 -----------------------------------------------%
\item If four piezoresistors from problem (6) are connected in a Wheatstone bridge configuration, as shown below, calculate $V_{out} = V_2 - V_1$ for all resistors experiencing a 0.1\% axial strain.
\begin{figure}[h]
\centering
\begin{subfigure}[b]{0.45\textwidth}
	\centering
	\begin{circuitikz}[scale=0.9]\draw
		(0,1) node[ground] (gnd) {}
			to[vR, l_=$R_{n1}$, o-*] (0,3)
			to[vR, l_=$R_{p1}$, *-o]  (0,5)
			node[anchor=south] {$+10V$};
		\draw (0,3) to[short, *-*] (2,3)
			node[anchor=south] {$V_{1}$};
		\draw (6,1) node[ground] (gnd) {}
			to[vR, l_=$R_{p2}$, o-*] (6,3)
			to[vR, l_=$R_{n2}$, *-o]  (6,5)
			node[anchor=south] {$+10V$};
		\draw (6,3) to[short, *-*] (4,3)
			node[anchor=south] {$V_{2}$};
	\end{circuitikz}
	\caption{The Circuit}
\end{subfigure}
\begin{subtable}[b]{0.45\textwidth}
	\centering
	\begin{tabular}{ l }
		$R_{p1} = R_{p2} = R_p + (GF_p)(\varepsilon_l)(R_p)$\\
		$R_{p1} = R_{p2} = 1k + 30(.001)(1k) = 1030\Omega$\\
		$R_{n1} = R_{n2} = R_n + (GF_n)(\varepsilon_l)(R_n)$\\
		$R_{n1} = R_{n2} = 1k - 30(.001)(1k) = 970\Omega$\\
	\end{tabular}
	\caption{Calculate Resistances.}
\end{subtable}
\begin{subtable}[b]{0.30\textwidth}
	\centering
	\begin{tabular}{ l }
		$V_{1} = \frac{10 R_{n1}}{R_{n1} + R_{p1}}$\\
		$V_{1} = \frac{10 (970)}{970 + 1030}$\\
		$V_{1} = 4.85V$\\
	\end{tabular}
	\caption{Calculate $V_{1}$.}
\end{subtable}
\begin{subtable}[b]{0.30\textwidth}
	\centering
	\begin{tabular}{ l }
		$V_{2} = \frac{10 R_{p2}}{R_{n2} + R_{p2}}$\\
		$V_{2} = \frac{10 (1030)}{970 + 1030}$\\
		$V_{2} = 5.15V$\\
	\end{tabular}
	\caption{Calculate $V_{2}$.}
\end{subtable}
\begin{subtable}[b]{0.30\textwidth}
	\centering
	\begin{tabular}{ l }
		$V_{out} = V_2 - V_1$\\
		$V_{out} = 0.3V$\\
	\end{tabular}
	\caption{Calculate $V_{out}$.}
\end{subtable}



\end{figure}

\end{enumerate}


%  \begin{figure}[htbp]
%   \centering
%   \includegraphics[width=4.0in,keepaspectratio]{E-Field}
%   \caption{\small{ The E-Field pattern produced by the initial code. }}
%   \label{fig:E-Field}
%   \end{figure}
%  \begin{figure}[htbp]
%   \centering
%   \includegraphics[width=4.0in,keepaspectratio]{Power}
%   \caption{\small{ The normalized power pattern of the system.  }}
%   \label{fig:Power}
%   \end{figure}

\label{end}\end{document}


