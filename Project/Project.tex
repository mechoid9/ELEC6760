% -----------  Document Initializations ------------ %
\documentclass[12pt]{article}
\usepackage[top=1in, bottom=1in, left=1.25in, right=1.25in]{geometry}
% \usepackage{wordlike}
\usepackage{txfonts}
\pagenumbering{gobble}
% -------------------------------------------------- %

% -----------  Document Declarations --------------- %
\title{ELEC 6760 Project, Spring 2013: ADXL278 - Dual-Axis, High-\emph{g}, Accelerometers}
\date{2013.04.24}
\author{Brian Arnberg}
% -------------------------------------------------- %

% -----------  Document Content -------------------- %
\begin{document}\label{start}
\noindent\textbf{ELEC 6760 Project, Spring 2013: ADXL278 - Dual-Axis, High-\emph{g}, Accelerometers}\\
\\
Brian Arnberg\\

The ADXL278 is a dual-axis accelerometer capable of measuring high-\emph{g} accelerations. 
It is a low power device built on a single monolithic IC. Additionally, the voltage outputs 
the device are signal conditioned. It can measure dynamic or static acceleration. Also, 
its effective acceleration magnitudes are 35, 50, and 70 -\emph{g} in the x-axis and 
35, 50, and 35 -\emph{g} in the y-axis. The data sheet indicates that while the device
is primarily concerned with airbag applications (front/side), it can be used in a wide 
variety of other applications. This is because the device is designed to work in an 
automotive vehicle. Because it is designed for such use, it is temperature stable and 
``accurate over the automotive temperature range'' (pg 1). In addition to this built
in stability, the ADXL278 boasts a self-test feature that tests all of the relevant elements 
(electrical and mechanical) by exercising them. The MEMS sensor is an 8-terminal package. 

The data sheet lists specifications for three different models: AD22284, AD22285, and AD22286.  
For each model, the temperature range is from -40 C to +105 C. The non-linearity of each model
is typically 0.2\%. Each typically has a package alignment error of 1 Degree, with a 
sensor\--to\--sensor alignment error of 0.1 Degrees. The cross-axis sensitivity is -5\% to +5\%.
The resonant frequency is 24kHz. For the first model, the ratiometric sensitivity is 
55mV/\emph{g}; for the second, 38mV/\emph{g}, and for the third, 55mV/\emph{g}. 
The Zero\--\emph{g} Output Voltage offset for the first two models ranges from -150mV to +150mV 
(in both axes), the ranges for the third model are -100mV to +100mV in the X-axis and 
-150mV to +150mV in the Y-axis. The clock noise for each model is 5mV p-p. The frequency response
for -3dB Frequency is typically 400 Hz (for the 2-pole Bessel filter), while the frequency response
for the -3dB Frequency Drift is 2Hz (for the same 2-pole Bessel filter). The self-test has a 
logic high of 3.5V and a logic low of 1V, with a typical input resistance of 50kOhms. 

The data sheet lists some absolute maximum ratings for the ADXL278. The maximum allowable acceleration,
in any axis, powered or un-powered, is 4,000\emph{g}. The maximum Vs is -0.3V to +7.0V. For other pins, 
the maximum range is (COM-0.3V) to (Vs+0.3V). There is an indefinite output short-circuit duration
maximum rating. The maximum temperature range, for both operation and storage, is -65 C to +150 C. 
Operating the device outside any of these ranges, or above any of the maximum settings, is liable to 
permanently degrade the device. Additionally, this device is sensitive to Electrostatic Discharge. 

The chip has 8-terminals. Terminals 1, 7, and 8 are to be connected to VDD3, VDD, and VDD2, all of which
can range from 3.5V to 6V. Pins 2 and 6 serve as output pins for Yout and Xout, respectively. 
Pin 3 is the common pin, and Pin 4 is used to activate the self-test. Pin 5 is not to be connected. 

\end{document}\label{end}
% -------------------------------------------------- %
